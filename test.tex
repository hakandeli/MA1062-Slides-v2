\documentclass[english,notes]{beamer}
\usetheme[markdown]{IKRv2}

\title{Start of double line title\\Start of single line title}
\subtitle{Subtitle}
\date{Date of presentation}
\author{Name author1, name author2}
\email{Email addresses of authors}
\subject{<Subject Variable>}

\usepackage{lipsum}

\begin{document}

\newcommand{\testsize}[1]{#1 \string#1}
\begin{frame}{Font Size}
    \testsize{\tiny} (same as footnotesize)\par
    \testsize{\scriptsize} (same as footnotesize)\par
    \testsize{\footnotesize}\par
    \testsize{\small}\par
    \testsize{\normalsize}\par
    \testsize{\large}\par
    \testsize{\Large}\par
    \testsize{\LARGE}\par
    \testsize{\huge}\par
    \testsize{\Huge}
\end{frame}

\renewcommand{\testsize}[1]{#1 \string#1 \lipsum[66]}
\begin{frame}{Font Size}
    \testsize{\tiny}\par
    \testsize{\scriptsize}\par
    \testsize{\footnotesize}\par
    \testsize{\small}\par
    \testsize{\normalsize}\par
    \testsize{\large}
\end{frame}

\begin{frame}{Font Size}
    \testsize{\Large}
\end{frame}

\renewcommand{\testsize}[1]{#1 \string#1 \lipsum[66]\par #1 \string#1 \lipsum[66]}
\begin{frame}{Font Size}
    \testsize{\tiny}\par
    \testsize{\scriptsize}\par
    \testsize{\footnotesize}\par
    \testsize{\small}
\end{frame}

\renewcommand{\testsize}[1]{#1 \string#1 Nunc sed pede. Praesent vitae lectus. Praesent neque justo, vehicula eget, interdum id, facilisis et, nibh.
Phasellus at purus et libero lacinia dictum.\par #1 \string#1 Nunc sed pede. Praesent vitae lectus. Praesent neque justo, vehicula eget, interdum id, facilisis et, nibh.
Phasellus at purus et libero lacinia dictum.}
\begin{frame}{Font Size}
    \testsize{\normalsize}\par
    \testsize{\large}
\end{frame}

\begin{frame}{Font Size}
    \testsize{\Large}
\end{frame}

\begin{frame}{Font Size}
    \testsize{\LARGE}
\end{frame}

\makeatletter
\ikr@gridtrue
\makeatother

\begin{frame}{Presentation \& Printout}
\heading{Do}
\begin{itemize}
\item Always create PDFs using FrameMaker with page size A4 and scale setting 133\%.
\item Activate the \textbf{"Generate Acrobat Data"} button for proper full screen layout.
\item Activate the \textbf{"Convert CMYK Colors to RGB"} option in the PDF Setup menu in order
to create a PDF with correct color settings.
\item For presentations, use this PDF in acroread in full screen mode.
\item For printing, use this PDF and \textbf{print with kpdf/kprinter}.
\end{itemize}

\heading{Don't}
\begin{itemize}
\item Do not create PS files and postprocess them with psnup, mpage, etc.
    \begin{itemize}
    \item FM will mess up the Bounding Box settings and you might get clipped slides
    \item Postprocessing tools will use other algorithms than FM to convert colors into RGB \&
    CMYK. This can result in bad colors in your presentation and printout.
    \item Some tools (ps2ps) replace fonts with bitmaps. You get large files which do not scale well.
    \end{itemize}
\item Do not print your PDF with acroread. There currently is a printing problem of acroread
(only Linux) in combination with printer c1 in our infrastructure.
\item You can print directly from FM, but the resulting colors are not as in the PDF.
\end{itemize}
\end{frame}

\note{Everything you want}

\makeatletter
\ikr@gridfalse
\makeatother

\begin{frame}{Paragraph Formats}
\begin{itemize}
    \item I1L-indent 1 listing\\
    second line
    \item I1L-indent 1 listing\par
    second paragraph
    \item I1L-indent 1 listing
\end{itemize}
\end{frame}

\begin{frame}{Paragraph Formats}
I1L-indent 1 listing\\
second line

I1L-indent 1 listing\par
second paragraph

I1L-indent 1 listing
\end{frame}

\begin{frame}{Paragraph Formats}{Indent Formats}
\vspace{1ex}
\begin{itemize}
    \item I10-indent 1\\
    next line
    \begin{itemize}
        \item I20-indent 2\\
        next line
        \begin{itemize}
            \item I30-indent 3\\
            next line
        \end{itemize}
    \end{itemize}
\end{itemize}
\vspace{4ex}
\begin{itemize}
    \item I1L-indent 1 listing
    \begin{itemize}
        \item I2L-indent 2 listing
        \begin{itemize}
            \item I3L-indent 3 listing
        \end{itemize}
    \end{itemize}
\end{itemize}
\vspace{2.2ex}
\begin{enumerate}
    \item I1N-indent 1 num
    \begin{enumerate}
        \item I2N-indent 2 num
        \begin{enumerate}
            \item I3N-indent 3 num
            \item I3N-indent 3 num
        \end{enumerate}
    \end{enumerate}
\end{enumerate}
\end{frame}

\begin{frame}{Arrows}
\begin{arrowlist}
    \item A
    \item B
    \begin{arrowlist}
        \item A
        \begin{arrowlist}
            \item A
        \end{arrowlist}
    \end{arrowlist}
\end{arrowlist}
\end{frame}

\begin{frame}{Arrows}
\conclusionarrow{} A

\conclusionarrow{} B
\end{frame}

\begin{frame}{Arrows}
A

B
\end{frame}

\begin{frame}{Arrows}
\begin{arrowlist}
    \item A
\end{arrowlist}
\begin{itemize}
    \item B
\end{itemize}
\end{frame}

\begin{frame}{Evaluation}{Compliance Ratio}
    \heading{Heading}
    This is normal text
    \heading{Heading}
    This is normal text

    $\int_0^{1^{\sum^{\sum^{\sum^{\sum}}}}}$\\
    a
\end{frame}

\begin{frame}{Title $\int_{0_{0_0}}^{1^{\sum^\sum}}$}
    \heading{Heading $\int_{0_{0_0}}^{1^{\sum^\sum}}$}
    This is normal text
    \heading{Heading}
    This is normal text

    $\int_0^{1^{\sum^\sum}}$\\
    a
\end{frame}

\begin{frame}{Title}
    \heading{Heading $\int_{0_{0_0}}^{1^{\sum^\sum}}$}
    This is normal text
    \heading{Heading}
    This is normal text

    $\int_0^{1^{\sum^\sum}}$\\
    a
\end{frame}

\begin{frame}{Title}
    \heading{Heading}
    This is normal text
    \heading{Heading}
    This is normal text

    $\int_0^{1^{\sum^\sum}}$\\
    a
\end{frame}

\begin{frame}{Title $\int_0^{1^{\sum^\sum}}$}
    \heading{Heading}
    This is normal text

    a\\
    a
\end{frame}

\begin{frame}{Title $\int_0^{1^{\sum^\sum}}$}{Subtitle $\int_0^{1^{\sum^\sum}}$}
    \heading{Heading}
    This is normal text
\end{frame}

\begin{frame}{Title}{Subtitle $\int_0^{1^{\sum^\sum}}$}
    \heading{Heading}
    This is normal text
\end{frame}

\begin{frame}{Title}{Subtitle}
    \heading{Heading}
    This is normal text
\end{frame}

\begin{frame}{Title}{Subtitle}
    This is normal text
\end{frame}

\begin{frame}{Title}
    \heading{Heading}
    This is normal text
\end{frame}

\begin{frame}{Title}
    This is normal text
\end{frame}

\begin{frame}{Title}{Subtitle}
Text

\heading{Heading}
This is normal text

\begin{itemize}
    \item bullet
    \item list
\end{itemize}

\begin{arrowlist}
    \item arrow
    \item list
\end{arrowlist}

\heading{Heading}
\end{frame}

\makeatletter
\ikr@gridtrue
\makeatother

\begin{frame}{Title}{Subtitle}
Text

\heading{Heading}
This is normal text

bullet

list

arrow

list

\heading{Heading}
\end{frame}

\newcommand{\testbox}[2]{%
    \begin{textblock}{#1}(#2)
        |\{#1\}(#2)\hfill |
    \end{textblock}
}

\begin{frame}{Positioning Objects}
\testbox{2}{0,1}
\testbox{3}{1,4}
\testbox{4}{2,5}
\testbox{5}{8,6}
\testbox{5}{8,8}
\end{frame}

\makeatletter
\ikr@gridfalse
\makeatother

\begin{frame}{Columns n}
A

B
\end{frame}

\begin{frame}{Columns}
    \begin{cols}
        \col{0.5\textwidth}
            A

            B
        \col{0.5\textwidth}
            B
    \end{cols}
\end{frame}

\begin{frame}{Columns n}
\heading{A}

B\\
C

D
\heading{E}
F
\end{frame}

\begin{frame}{Columns 4}
    \begin{cols}
        \col{0.5\textwidth}
            \heading{A}

            B\\
            C

            D
            \heading{E}
            F
    \end{cols}
\end{frame}

\begin{frame}{Columns}
    \begin{cols}
        \col{0.5\textwidth}
            \heading{A}

            B\\
            C

            D
            \heading{E}
            F
        \col{0.5\textwidth}
            B

            C
    \end{cols}
\end{frame}

\begin{frame}{Columns}
    \begin{cols}
        \col{0.5\textwidth}
            \heading{A}

            B\\
            C

            D
            \heading{E}
            F
        \col{0.5\textwidth}
            B

            C
    \end{cols}
    \begin{cols}
        \col{0.5\textwidth}
            F
        \col{0.5\textwidth}
            B

            C
    \end{cols}
\end{frame}

\begin{frame}{Columns n}
    \heading{A}

    B\\
    C

    D
    \heading{E}
    F

    F
\end{frame}

\begin{frame}{Columns}
    \begin{cols}
        \col{0.5\textwidth}
            \heading{A}

            B\\
            C

            D
            \heading{E}
            F
        \col{0.5\textwidth}
            %\vspace{0pt}
            \Huge B

            C
    \end{cols}
\end{frame}

\begin{frame}{Columns}
    \begin{cols}
        \col{0.5\textwidth}
            \heading{A}

            B\\
            C

            D
            \heading{E}
            F
        \col{0.5\textwidth}
            B

            C
    \end{cols}
\end{frame}

\begin{frame}{Columns}
    \begin{cols}
        \col{0.5\textwidth}
            \heading{A}

            B\\
            C

            D
            \heading{E}
            F
        \col{0.5\textwidth}
            B

            C
    \end{cols}
\end{frame}

\begin{frame}{Columns}
    \begin{cols}
        \col<2->{0.5\textwidth}
            \heading{A}

            B\\
            C

            D
            \heading{E}
            F
        \col{0.5\textwidth}
            B

            C
    \end{cols}
\end{frame}

\begin{frame}{Columns}
    \begin{cols}
        \col{0.5\textwidth}
            \heading{A}

            B\\
            C

            D
            \heading{E}
            F
        \col{0.5\textwidth}
            \vspace{-\topskip}
            \Huge B

            C
    \end{cols}
\end{frame}

\begin{frame}{Columns}
    Th
    \begin{cols}
        \col{0.5\textwidth}
            \heading{A}

            B\\
            C

            D
            \heading{E}
            F
    \end{cols}
\end{frame}

\begin{frame}{Columns}
    Tp
    \begin{cols}
        \col{0.5\textwidth}
            \heading{A}

            B\\
            C

            D
            \heading{E}
            F
    \end{cols}
\end{frame}

\begin{frame}{Tabs}
a \tab b

a \tabs 2 b

a \tabs 3 b
\end{frame}

\begin{frame}{Chars}
    > <
    
    $> <$
\end{frame}

\begin{frame}{Textblock top}
\begin{textblock}{5}(7,7)
    Text
\end{textblock}

A
\end{frame}

\begin{frame}{Textblock bottom}
A
\begin{textblock}{5}(7,7)
    Text
\end{textblock}
\end{frame}

\begin{frame}{Textblock top}
    
\begin{textblock}{5}(7,7)
    Text
\end{textblock}

# A
\end{frame}

\begin{frame}{Textblock bottom}
# A
\begin{textblock}{5}(7,7)
    Text
\end{textblock}
\end{frame}

\begin{frame}{Tables}
\begin{tblr}{lll}
    Ag & Bg & Cgp \\
    a & b & c \\
    d & e & f
\end{tblr}

\begin{tblr}{width=0.8\linewidth,colspec={X[5ex,l]X[3,l]X[-1,r]X[r]}}
    Ag & Bg $\sum_1^2$ & Cgp & D \\
    a & b & c & d \\
    d & e & f & e
\end{tblr}

\begin{tblr}{lll}
    A & B & C \\
    a & b & c \\
    d & e & f
\end{tblr}
\end{frame}

\begin{frame}{Markdown (Ref)}
\heading{Test}
\begin{itemize}
    \item A
    \begin{itemize}
        \item a
        \item b
    \end{itemize}
    \item B
\end{itemize}
\begin{arrowlist}
    \item A
    \item B
\end{arrowlist}
\end{frame}

\begin{frame}{Markdown}
# Test % comment
* A
 * a
 * b
* B

-> A
-> B

Result => Good

\end{frame}

\disablemarkdown
\begin{frame}{Markdown}
* A
* B

\end{frame}
\enablemarkdown

\begin{frame}{Overlay Pic}
\begin{cols}
    \col{0.45\textwidth}
        \heading{Column 1}
        This is text in the first column
    \col{0.25\textwidth}
        \pic<2->[width=\textwidth]{example-image}[caption=Test,captionpos=b]
    \end{cols}
\end{frame}

\begin{frame}{Centered Pic}
AAAAAAAAAAAAAAAAAAAAAAAAAAAAAAAAAAAAAAAAAAA
\pic[width=2cm]{example-image}[pos=c]
AAAAAAAAAAAAAAAAAAAAAAAAAAAAAAAAAAAAAAAAAAA

AAAAAAAAAAAAAAAAAAAAAAAAAAAAAAAAAAAAAAAAAAA
\begin{center}
    Pic
\end{center}
AAAAAAAAAAAAAAAAAAAAAAAAAAAAAAAAAAAAAAAAAAA
\end{frame}

\begin{frame}{Title}

% empty line required to distinguish from subtitle
{\centering A \par}
\end{frame}

\begin{frame}{Title}
\begin{cols}
    \col{0.25\textwidth}
    Test
    \col{0.25\textwidth}
    \pic[width=\textwidth]{example-image}[caption=Test]
    \col{0.25\textwidth}
    \pic[width=\textwidth]{example-image}[caption=Test,captionpos=b]
\end{cols}
\end{frame}

\begin{frame}{~}
\end{frame}

\end{document}
